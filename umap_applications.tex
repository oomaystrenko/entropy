\begin{itemize}
	\item \textbf{Энтропийное кодирование}
	
	Как говорилось ранее, энтропия показывает наименьшее среднее число бит, необходимое для кодирования некоторой информации. Данное свойство используется, как ни странно, при кодировании информации.
	
	Например, код Шеннона-Фано. С целью минимизации энтропии и, соответственно, оптимизации кода элементы с большой вероятностью появления кодируются меньшим числом символом. Таким образом, производится сжатие объема информации, что позволяет передавать большее количество информации, затрачивая меньший объем памяти.
	
	\item \textbf{Построение решающих деревьев}
	
	Решающие деревья - метод, использующийся в машинном обучении и работающий по принципу принятия решений человеком. Каждое ветвление представляет собой разделение выборки на 2 части по порогу некоторого признака. Например, признак - длина, пороговое значение -  2,5. Все объекты, длина которых превышает 2,5, отделяются от объектов с длиной меньше 2,5 и дальнейший анализ проходят отдельно.
	
	В данном методе расчет энтропии помогает определить оптимальный порог для каждого узла решения. А именно, подбирается такое разделение выборки, при котором сумма энтропий получившихся выборок минимальна среди возможных вариантов разбиений.
	
	Это позволяет получать после разбиения выборки, содержащие наименее разнообразные по содержанию классов. Соответственно, признак и пороговое значение подбираются наиболее оптимально - алгоритм успешно отделяет объекты, принадлежащие одному классу.
	
	\item \textbf{Применение в алгоритмах t-SNE и UMAP}
	
	В анализе данных часто возникает необходимость в снижении размерности, и в таких случаях на помощь приходят знания об энтропии, изученной в курсе теории вероятностей. Речь, конечно, идет не об энтропии как таковой, а об алгоритмах, которые базируются на теории.
	
	При создании пространства меньшей размерности, t-SNE и UMAP используют кросс-энтропию как показатель эффективности перенесения свойств объектов. Чем меньше кросс-энтропия, тем ближе к истинному оказалось подобранное распределение.
	
	
\end{itemize}