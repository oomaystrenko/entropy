Пример. В КЭБ717 учатся три прекрасные девушки (наши объекты): Даша, Яна и Лиза. Даша -- блондинка с голубыми глазами, у Яны --- темные волосы и карие глаза, у Лизы --- русые волосы и карие глаза. Попробуем отобразить объекты в одномерном пространстве, то есть с одним признаком.

Решение. Для начала приведем признаки к числовому виду. Первый признак -- цвет глаз: карие(1), голубые(0). Второй признак -- цвет волос: светлые(0), русые(1), темные(2). Теперь у нас есть три объекта:

\begin{center}
	\begin{tabular}{|c||c|c|}
		\hline
		 & Цвет глаз & Цвет волос\\
		\hline
		\hline
		D & 0 & 0\\
		\hline
		Y & 1 & 2\\
		\hline
		L & 1 & 1 \\
		\hline 
	\end{tabular}
\end{center}

Теперь найдем расстояния между объектами. По умолчанию, в UMAP стоит евклидова метрика. В нашем случае она также подходит, так как наши признаки могут быть интерпретированы как координаты точек в пространстве:

\begin{center}
	\begin{tabular}{|c||c|c|c|}
		\hline
		& D & Y & L\\
		\hline
		\hline
		D & 0 & 2.24 & 1.4\\
		\hline
		Y & 2.24 & 0 & 1\\
		\hline
		L & 1.4 & 1 & 0\\
		\hline 
	\end{tabular}
\end{center}