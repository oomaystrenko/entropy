Тут можно найти ответы и решения ко всем задачам, которые были представлены выше. Переходите к этому разделу, только если уже всё решили и хотите проверить себя :)

\subsection*{\hyperref[sec:problem1]{Задача 1.}}\label{sec:sol_problem1}

Воспользуемся формулой стандартной энтропии. Вероятность того, что бабушка находится в какой-либо одной конкретной деревне из трех равна 1/3. Следовательно, энтропия местонахождения бабушки Елены в таком случае равна: 
\[H=-\sum\limits_{i=1}^3 \cfrac{1}{3}\log \cfrac{1}{3} \]

Результатом подсчёта данного выражения будет число 1.58. Это означает, что в среднем (при большом количестве повторений эксперимента) нам понадобится 1.58 вопросов, чтобы верно назвать деревню. При округлении до ближайшего целочисленного даёт значение два. Оно и верно! Смотрите, можно узнать, в какой деревне живёт бабушка на 100 \%  всего за два вопроса. Например, 1) Название деревни - это гласная буква? (если да, то ответ уже найден; если нет следует задать еще один вопрос); 2) Название деревни начинается на букву Б? (в любом случае, далее вы уже сможете ответить на вопрос правильно с вероятностью, равной единице).  \\
В целом можем сказать, что в среднем нам хватит двух вопросов, чтобы угадать деревню, а вот одного вопроса может хватить не всегда. \\

Ответ: 1.58


\subsection*{\hyperref[sec:problem2]{Задача 2.}}\label{sec:sol_problem2}

РЕШЕНИЕ И ОТВЕТ

\subsection*{\hyperref[sec:problem3]{Задача 3.}}\label{sec:sol_problem3}

Сюрприз! Вам предоставляется невероятная возможность самостоятельно проверить свои силы и решить следующие задачи. Варианты решения мы просим присылать в issues на открытый репозиторий на сайте github.com\footnote{Ссылка на репозиторий: \url{https://github.com/oomaystrenko/entropy}}. Первое правильное решение поощряется специальным ценным призом. 

Желаем удачи и с нетерпением ждем Ваших решений!

Мы предоставляем ответы к задачам 3-5.
\\

Ответ: 

\subsection*{\hyperref[sec:problem4]{Задача 4.}}\label{sec:sol_problem4}

Ответ: 

\subsection*{\hyperref[sec:problem5]{Задача 5.}}\label{sec:sol_problem5}

Ответ: 

\subsection*{\hyperref[sec:problem6]{Задача 6.}}\label{sec:sol_problem6}

РЕШЕНИЕ и ОТВЕТ

\subsection*{\hyperref[sec:problem7]{Задача 7.}}\label{sec:sol_problem7}

РЕШЕНИЕ и ОТВЕТ

\subsection*{\hyperref[sec:problem8]{Задача 8.}}\label{sec:sol_problem8}

Сюрприз! Вам предоставляется невероятная возможность самостоятельно проверить свои силы и решить следующие задачи. Варианты решения мы просим присылать в issues на открытый репозиторий на сайте github.com. Первое правильное решение поощряется специальным ценным призом. 

Желаем удачи и с нетерпением ждем Ваших решений!

Мы предоставляем ответы к задачам 8-10.
\\

Ответ: 

\subsection*{\hyperref[sec:problem9]{Задача 9.}}\label{sec:sol_problem9}

Ответ: 

\subsection*{\hyperref[sec:problem10]{Задача 10.}}\label{sec:sol_problem10}

Ответ: 


