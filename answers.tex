Тут можно найти ответы и решения ко всем задачам, которые были представлены выше. Переходите к этому разделу, только если уже всё решили и хотите проверить себя :)

\subsection*{\hyperref[sec:problem1]{Задача 1.}}\label{sec:sol_problem1}
Теперь подсчитаем количество возможных вариантов размещения трех агентов: Шапки, бабули и волка. Общее число размещений каждого из трех агентов в трех местах равно 27. 
Далее, чтобы узнать вероятность хорошего исхода, необходимо знать, во скольких случаях бабушка не получает свои пирожки (то есть Шапка и волк оказываются в одной деревне вместе). Они могут оказаться в одной деревне только вдвоём или с бабулей, но и том, и в другом случае сказка не имеет счастливого конца. Таких вариантов девять. Тогда вариантов, в которых бабуля наслаждается пирожками 18.
Энтропия равна:
\[H=-\sum\limits_{i=1}^{18}  \cfrac{1}{18}\log \cfrac{1}{18} \]

Результатом подсчёта данного выражения будет число 4.17. Это означает, что в среднем (при большом количестве повторений эксперимента) ей будет достаточно 4.17 вопросов, чтобы угадать деревню и доставить пирожки. При округлении до ближайшего целочисленного дает значение пять. Это означает, что в среднем за пять вопросов возможно узнать, получила ли бабушка свои пирожки и в какой деревне она находится, а вот за четыре вопроса это будет сделать сложнее. 

Ответ: а) 18 вариантов; б) 4.17


\subsection*{\hyperref[sec:problem2]{Задача 2.}}\label{sec:sol_problem2}

Всего вариантов шесть (три деревни и Волка не было, волк был, но его поймали). Так как все три деревни равновероятны, рассмотрим одну из них. Вероятность того, что волка не было: \[\cfrac{3-N}{3} \]
Вероятность того, что он был и его поймали: \[\cfrac{N}{3}\cdot \cfrac{1}{3} \]
Одно из событий наступило гарантированно: 
\[\cfrac{3-N}{3} + \cfrac{N}{9} = \cfrac{9-2N}{9} \]
Тогда вeроятность $\mathbb{P}$(Волк был в деревне А и его поймали):
\[\cfrac{N}{9}\cdot \cfrac{9}{(9-2N)\cdot3} =\cfrac{N}{(9-2N)\cdot3} \] 
$\mathbb{P}$(Волка не было в деревне А):
\[\cfrac{(3-N)}{3}\cdot \cfrac{9}{(9-2N)\cdot3}=\cfrac{3-N}{9-2N}\]

Аналогично для других деревень.

Получаем энтропию:
\[H=3\cdot \cfrac{N}{(9-2N)\cdot3}\cdot \log \cfrac{(9-2N)\cdot3}{N}+3\cdot\cfrac{3-N}{9-2N}\cdot \log \cfrac{9-2N}{3-N}  \]

Ответ: см. в решении

\subsection*{\hyperref[sec:problem3]{Задача 3.}}\label{sec:sol_problem3}

Волк равновероятно находится в любой из деревень: в А, Б и В. Вероятность каждого из них равна 1/3. 
Бабушка живет в деревнях с разной вероятностью. Таким образом вероятность того, что они были в деревне А равна 1/6, так как мы перемножили 1/2 и 1/3. Аналогично, вероятность того, что они были в деревнях Б и В равна 1/12. 

Итого совместная вероятность:
\[H=-  \left(\cfrac{1}{6}\log \cfrac{1}{6} + \cfrac{1}{12}\log \cfrac{1}{12} + \cfrac{1}{12}\log \cfrac{1}{12} \right) \] 

Ответ: $\cfrac{1}{2} + \cfrac{1}{3}\cdot\log3$

\subsection*{\hyperref[sec:problem4]{Задача 4.}}\label{sec:sol_problem4}

Сюрприз! Вам предоставляется невероятная возможность самостоятельно проверить свои силы и решить такую задачу. Варианты решения мы просим присылать в issues на открытый репозиторий на сайте github.com\footnote{Ссылка на репозиторий: \url{https://github.com/oomaystrenko/entropy}}. Первое правильное решение поощряется специальным ценным призом. 

Желаем удачи и с нетерпением ждем Ваших решений!

\subsection*{\hyperref[sec:problem5]{Задача 5.}}\label{sec:sol_problem5}
Съесть пирожок с капустой можно 20 вариантами (а шапочка считает, что 10).
Итого, 10 пирожков с капустой и КШ считает, что они с капустой, а 10 с капустой и КШ не считает их с капустой. 
Тогда:
\[H=10\cdot\cfrac{1}{20}\cdot\log10=\cfrac{\log10}{2}\]

А решение второго пункта мы также принимаем в issues на открытый репозиторий \url{github.com}. Первый правильно решивший задачу получит ценный приз от терверисток из группы БЭК171! :)

Ответ: а) $\frac{\log10}{2}$, б) ждем Ваших решений!

\subsection*{\hyperref[sec:problem6]{Задача 6.}}\label{sec:sol_problem6}

$X$ --- расстояние, которое прошла Шапка:
\[f(x)=\cfrac{1}{B}\]
\[H(x)=-\int\limits_{0}^{B}\frac{1}{B}\log_2\left(\frac{1}{B}\right)dx=\frac{1}{B}\cdot \log_2 B\cdot B=\log_2 B\]

Ответ: $\log_2 B$

\subsection*{\hyperref[sec:problem7]{Задача 7.}}\label{sec:sol_problem7}

$Y$ --- расстояние, которое прошла Шапка:
\[f(x, y)=\cfrac{1}{B}\cdot \cfrac{1}{x}\Rightarrow f_{Y|X}(y)=\cfrac{1}{B} \]
\[H(Y|X)=-\int\limits_0^B \frac{1}{B}\cdot \frac{1}{x}\cdot \log_2 \left(\frac{1}{B}\right)dy=\frac{1}{x}\log_2 B \]

Ответ: $\cfrac{\log_2 B}{x}$


