Тут можно найти ответы и решения ко всем задачам, которые были представлены выше. Переходите к этому разделу, только если уже всё решили и хотите проверить себя :)

\subsection*{\hyperref[sec:problem1]{Задача 1.}}
\label{sec:sol_problem1}\
\\

Так как деревни всего 3 и вероятность Красной Шапочки прийти в любую из них одинаковая, то примем параметр p = 1/3. Тогда, следуя формуле выше, получаем, что:
\[H=-\sum\limits_{i=1}^3 \cfrac{1}{3}\log \cfrac{1}{3} \]

Результатом подсчёта данного выражения будет число 1.58, что при округлении до ближайшего целочисленного даёт значение 2. Оно и верно! Смотрите, можно узнать, в какой деревне живёт бабушка на 100 \%  всего за два вопроса. 
Например, 1) Название деревни - это гласная буква? (если да, то ответ уже найден; если нет следует задать еще один вопрос); 2) Название деревни начинается на букву Б? (в любом случае, далее вы уже сможете ответить на вопрос правильно с вероятностью, равной единице). \\

Ответ: 2
\subsection*{\hyperref[sec:problem2]{Задача 2.}}
\label{sec:sol_problem2}\
\\

Теперь давайте подсчитаем количество возможных вариантов размещения трех агентов: Шапки, бабули и волка. Очевидно, что общее число размещений каждого из 3 агентов в 3 местах равно 27. 
Далее, чтобы узнать вероятность хорошего исхода, необходимо узнать, во скольких случаях бабушка не получает свои пирожки (то есть Шапка и волк оказываются в одной деревне вместе). Причём заметим, они могут оказаться в одной деревне только вдвоём или с бабулей, однако и том, и в другом случае сказка не имеет счастливого конца. Таких вариантов всего 9. Тогда вариантов, в которых бабуля наслаждается пирожками 18 (получено как 27-9).
Тогда подсчитать энтропию в данном случае можно по следующей формуле:
\[H=-\sum\limits_{i=1}^{18}  \cfrac{1}{18}\log \cfrac{1}{18} \]
Результатом подсчёта данного выражения будет число 4.17, что при округлении до ближайшего целочисленного дает значение 5. Это означает, что за 5 вопросов возможно узнать, получила ли бабушка свои пирожки и в какой деревне она находится. \\ 

Ответ: 5

\subsection*{\hyperref[sec:problem3]{Задача 3.}}
\label{sec:sol_problem3} \
\\ 

В данном случае наших героев стало уже четверо! Теперь количесвто возможных расположений увеличилось до 81, сотвественно (легко дойти до этого самостоятельно). Теперь выясним количество благоприятных исходов, то есть, когда Шапка, волк и охотники оказались в одной деревне. Для этого нам проще выяснить обратное событие, то есть когда Шапка и волк оказались в одной деревне вместе (как с бабулей, так и без нее), а охотники в это же время были в другом месте. Например, пусть Шапка и волк находятся вместе в 1 деревне, тогда возможных вариантов расположения охотников и бабушки, при которых Шапка не выживает и бабуля не получает свои пирожки - 6 штук. Тогда, распространив случай  на три деревни, получаем, что всего печальных исходов 18. Тогда благоприятных исходов 63. Тогда получить ответ можно, воспользовавшишь формулой: 
\[H=-\sum\limits_{i=1}^{63}  \cfrac{1}{63}\log \cfrac{1}{63} \]
Результатом подсчёта данного выражения будет число 5.9, что при округлении до ближайшего целочисленного дает значение 6. \\

Ответ: 6
\subsection*{\hyperref[sec:problem4]{Задача 4.}}
\label{sec:sol_problem4}

\subsection*{\hyperref[sec:problem5]{Задача 5.}}\label{sec:sol_problem5}

\subsection*{\hyperref[sec:problem6]{Задача 6.}}\label{sec:sol_problem6}

\subsection*{\hyperref[sec:problem7]{Задача 7.}}\label{sec:sol_problem7}

\subsection*{\hyperref[sec:problem8]{Задача 8.}}\label{sec:sol_problem8}