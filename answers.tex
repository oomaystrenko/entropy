Тут можно найти ответы и решения ко всем задачам, которые были представлены выше. Переходите к этому разделу, только если уже всё решили и хотите проверить себя :)

\subsection*{\hyperref[sec:problem1]{Задача 1.}}
\label{sec:sol_problem1}\
\\

Так как деревни всего 3 и вероятность Красной Шапочки прийти в любую из них одинаковая, то примем параметр p = 1/3. Получаем, что:
\[H=-\sum\limits_{i=1}^3 \cfrac{1}{3}\log \cfrac{1}{3} \]

Результатом подсчёта данного выражения будет число 1.58. Это означает, что в среднем (при большом количестве повторений эксперимента) нам понадобится 1.58 вопросов, чтобы верно назвать деревню. При округлении до ближайшего целочисленного даёт значение 2. Оно и верно! Смотрите, можно узнать, в какой деревне живёт бабушка на 100 \%  всего за два вопроса. Например, 1) Название деревни - это гласная буква? (если да, то ответ уже найден; если нет следует задать еще один вопрос); 2) Название деревни начинается на букву Б? (в любом случае, далее вы уже сможете ответить на вопрос правильно с вероятностью, равной единице).  \\
В целом можем сказать, что в среднем нам хватит двух вопросов, чтобы угадать деревню, а вот одного вопроса может хватить не всегда. \\

Ответ: 1.58
\subsection*{\hyperref[sec:problem2]{Задача 2.}}
\label{sec:sol_problem2}\
\\

Теперь подсчитаем количество возможных вариантов размещения трех агентов: Шапки, бабули и волка. Общее число размещений каждого из 3 агентов в 3 местах равно 27. 
Далее, чтобы узнать вероятность хорошего исхода, необходимо знать, во скольких случаях бабушка не получает свои пирожки (то есть Шапка и волк оказываются в одной деревне вместе). Они могут оказаться в одной деревне только вдвоём или с бабулей, но и том, и в другом случае сказка не имеет счастливого конца. Таких вариантов 9. Тогда вариантов, в которых бабуля наслаждается пирожками 18.
Энтропия равна:
\[H=-\sum\limits_{i=1}^{18}  \cfrac{1}{18}\log \cfrac{1}{18} \]
Результатом подсчёта данного выражения будет число 4.17. Это означает, что в среднем ей будет достатчоно 4.17 вопросов, чтобы угадать деревню и доставить пирожки. При округлении до ближайшего целочисленного дает значение 5. Это означает, что в среднем за 5 вопросов возможно узнать, получила ли бабушка свои пирожки и в какой деревне она находится, а вот за 4 вопроса это будет сделать сложнее. \\ 

Ответ: 4.17

\subsection*{\hyperref[sec:problem3]{Задача 3.}}
\label{sec:sol_problem3} \
\\ 

Наших героев стало уже четверо! Теперь количесвто возможных расположений увеличилось до 81. Выясним количество благоприятных исходов, то есть, когда Шапка, волк и охотники оказались в одной деревне. Для этого нам проще выяснить обратное событие, то есть когда Шапка и волк оказались в одной деревне вместе (как с бабулей, так и без нее), а охотники в это же время были в другом месте. Например, пусть Шапка и волк находятся вместе в 1 деревне, тогда возможных вариантов расположения охотников и бабушки, при которых Шапка не выживает и бабуля не получает свои пирожки - 6 штук. Тогда, распространив случай  на три деревни, получаем, что всего печальных исходов 18. Благоприятных исходов 63. Энтропия в этом случае: 
\[H=-\sum\limits_{i=1}^{63}  \cfrac{1}{63}\log \cfrac{1}{63} \]
Результатом подсчёта данного выражения будет число 5.9. \\

Ответ: 5.9

\subsection*{\hyperref[sec:problem4]{Задача 4.}}
\label{sec:sol_problem4}

\subsection*{\hyperref[sec:problem5]{Задача 5.}}\label{sec:sol_problem5}

\subsection*{\hyperref[sec:problem6]{Задача 6.}}\label{sec:sol_problem6}

\subsection*{\hyperref[sec:problem7]{Задача 7.}}\label{sec:sol_problem7}

\subsection*{\hyperref[sec:problem8]{Задача 8.}}\label{sec:sol_problem8}