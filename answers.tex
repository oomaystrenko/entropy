Тут можно найти ответы и решения ко всем задачам, которые были представлены выше. Переходите к этому разделу, только если уже всё решили и хотите проверить себя :)

\subsection*{\hyperref[sec:problem1]{Задача 1.}}
\label{sec:sol_problem1}
Так как деревни всего 3 и вероятность Красной Шапочки прийти в любую из них одинаковая, то примем параметр p = 1/3. Тогда, следуя формуле выше, получаем, что:
\[H=-\sum\limits_{i=1}^3 1/3\log 1/3 \]

Результатом подсчета данного выражения будет число 1,58, что при округлении до ближайшего целочисленного дает значение 2. Оно и верно! Смотрите, можно узнать, в какой деревне живет бабушка на 100 \%  всего за два вопроса. 
Например, 1) Название деревни - это гласная буква? (если да, то ответ уже найден; если нет следует задать еще один вопрос); 2) Название деревни начинается на букву Б? (в любом случае, далее вы уже сможете ответить на вопрос правильном с вероятностью, равной единице).

Ответ: 2
